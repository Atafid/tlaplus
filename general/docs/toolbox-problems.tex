\documentclass{article}
\usepackage{tla}

%%%%%%%%%%%%%%%%%%%%%%%%%%%%%%%%%%%%%%%%%%%%%%%%%%%%%%%%%%%%%%%%%%%
%                 MODIFICATION DATE                               %
%%%%%%%%%%%%%%%%%%%%%%%%%%%%%%%%%%%%%%%%%%%%%%%%%%%%%%%%%%%%%%%%%%%
%                                                                 %
% Defines \moddate to expand to modification date such as         %
%                                                                 %
%    5 Aug 1991                                                   %
%                                                                 %
% and \prdate to print it in a large box.  Assumes editor         %
% updates modification date in standard SRC Gnu Emacs style.      %
% (should work for any user name).                                %
%%%%%%%%%%%%%%%%%%%%%%%%%%%%%%%%%%%%%%%%%%%%%%%%%%%%%%%%%%%%%%%%%%%
\def\ypmd{%                                                       %
%                                                                 %
%                                                                 %
  Last modified on Fri 30 October 2009 at  8:01:22 PST by lamport         %
  endypmd}                                                        %
%                                                                 %
%%%%%%%%%%%%%%%%%%%%%%%%%%%%%%%%%%%%%%%%%%%%%%%%%%%%%%%%%%%%%%%%%%%
\newcommand{\moddate}{\expandafter\xpmd\ypmd}                     %
\def\xpmd Last modified                                           %
on #1 #2 #3 #4 at #5:#6:#7 #8 by #9 endypmd{#2 #3 #4}                %
\newcommand{\prdate}{\noindent\fbox{\Large\moddate}}              %
%%%%%%%%%%%%%%%%%%%%%%%%%%%%%%%%%%%%%%%%%%%%%%%%%%%%%%%%%%%%%%%%%%%

\newcommand{\TLA}[1]{TLA$^{+}$}
\newcommand{\mytt}{\normalfont\ttfamily}
\newcommand\bs{\char '134 }  % A backslash character for \tt font

\title{List of Known Toolbox Problems}
% \author{Leslie Lamport}
\date{\moddate}

\begin{document}
\maketitle

\noindent These are all the known bugs and missing features
of the \TLA+ Toolbox reported as of
\moddate.  


\begin{description}

\item[1. New Feature] \mbox{}\\
Add way to tell if the spec on which a model was last run is the same as
the current spec.
   
\item[5. New Feature (Locking Models)] \mbox{}\\
%
TLC might be running for weeks, and the user in the meantime changed
the spec.  When TLC stops running, the model gets revalidated and
stuff that was in the model configuration gets lost.  Here is a
possible solution: Add the notion of a model being LOCKED, meaning
that it cannot be changed in any way, meaning that it is \emph{read only} 
and it doesn't change to conform to the current spec. In
particular, it conforms to the copy of the spec that's saved
with the model.  Running the spec locks the model.  There will also be
a menu item that the use can select to lock the spec himself.  (This
will allow the user to protect himself from accidentally modifying or
re-running a spec by mistake---perhaps losing a trace that took weeks
for TLC to produce.)  We must decide what should happen to the lock
state when a TLC run completes.  Should the user have to explicitly
unlock it?  I suggest that it be unlocked when the run completes if
and only if that run did not take too long---where ``too long'' is a
user-settable preference.

\item[9. Minor Missing Features (TLC Model)] \mbox{}\\
%
The user cannot now declare the same model value in two different
constant instantiations or in a constant instantiation and a
definition override.  Moreover, he cannot use a model value declared
on the advanced page as a model value in a constant instantiation.
These restrictions should be eliminated.

\item[12. Minor Bug (Parsing Error View)] \mbox{}\\
%
If the spec is reparsed after a parsing error, the Parsing Error
view is displayed, and no errors are found, then the Parsing Error view
continues to contain the previous error.  The view has to be closed to
make that error disappear.
 

\item[13. Bug (PlusCal Translator Errors)] \mbox{}\\
%
Suppose the user starts to write a PlusCal algorithm, translates it
and gets a PlusCal error, and then decides to delete the PlusCal code
and just write the spec in \TLA+.  There's no way for him to get rid of
the PlusCal error message in the Parse Errors window.  In fact, even
if I run the translator on another module, the PlusCal error remains
in the Parse Error window---whether or not the other translation
produces an error.  (The only way to get rid of it is to go back to
the original module, correct the PlusCall error, and re-run the
translator.)  The best thing to do is probably to create a separate
PlusCal Error Window.  Running the translator should first clear this
window and remove all PlusCal error markers in the module on which the
translator is run.  (I believe that if running the PlusCal translator
creates a PlusCal error, then the file is not re-parsed, so only the
PlusCal error window would need to be popped up.)

\item[15. Possible Change (Handling Unparsed Spec)] \mbox{}\\
%
We need to re-think what the user should be able to do to the model
when the spec is unparsed; and see that anything he shouldn't be able
to do is disabled and that anything he should be able to do works.

\item[17. Minor Bug (Definition Override)] \mbox{}\\
%
The Definitions Override menu allows the user to override the same
definition multiple times. 

\item[18. Significant Bug (parsing large specs)] \mbox{}\\
%
For a really big spec (the one we tried had dozens of files and over
7000 lines of \TLA+), after the spec is first parsed the parse spec
command gets disabled.  However, parse module file still works.  

\item[19. Minor Missing Feature (Exit)] \mbox{}\\
%
Closing the toolbox simply kills any running TLC. Instead, it should
warn the user that this will kill TLC and give him the option of
canceling. 

\item[20. Missing Feature (PlusCal)] \mbox{}\\
%
PlusCal translation needs to be examined.  Here are some things that
need to be fixed:
\begin{itemize}
\item There should be a more user-friendly way to specify translation
   options.  This involves:
   (a) Providing check-boxes or radio buttons for the common options.
   (b) Finding a more convenient way to add them than the current
       one (right-clicking in the Toolbox Explorer) and selecting
       properties.
\item Translation options are now attached to the spec; they should
   be attached to an individual module.  (It's possible for a
   single spec to have PlusCal algorithms in two different 
   modules.)

\end{itemize}
   

\item[21. Tuning Needed (Preferences)] \mbox{}\\
%
We need to filter Preferences to eliminate things the user doesn't want
to see.

\item[22. Bug  (rename spec)] \mbox{}\\
%
The rename spec command doesn't do the right thing if a spec by that
name already exists.
      
\item[27. Minor Bug  (Parse Spec/Module Commands)] \mbox{}\\
%
If I select Parse spec from the file menu when the file is dirty, a
popup asks me if I want to save the resource.  If I say yes, it
doesn't save the file and parses the spec that's on disk.  If I select
Parse Module from the file menu when the module is dirty, it doesn't
ask me if I want to save it.  I expect it should.
 
\item[28. Minor Bug  (Reset Window Layout)] \mbox{}\\
%
If I select {\tt window/reset window layout} when the Welcome View
should be shown, it produces a blank window instead of the Welcome
View.

\item[29. Feature (Module Editor Keybindings)] \mbox{}\\
%
Here are some thoughts on enhancements to the Module Editor.  
\begin{itemize}
\item Add a
preference to choose an Emacs-like keybinding.  This requires
  adding new commands that appear in the General/Editors/Keys, and
  options that choose from among sets of bindings.
\item Add editing commands (and invent bindings) for comments.  
\end{itemize}
Note that \emph{toggle comments} is bound to Control-/, but that command
doesn't appear in the Preferences menu's keybinding options.

\item[30. Minor Bug (TLC Error View)] \mbox{}\\
%
When I close a spec that has the TLC Error window raised and then open
a new spec, the TLC Error window is raised again.  If that window
shows an error, then it shows the same error when the new spec is
opened.
 
\item[32. Minor Bug (What to parse when)] \mbox{}\\
%
The toolbox doesn't seem to be doing the right thing when deciding
what to parse when the user requests that a module be parsed.  This
needs to be re-examined and specified precisely, and then we need to
make sure that the implementation does what it's supposed to.

\item[33. Feature (TLC error reporting)] \mbox{}\\
%
When TLC returns an error stack trace, make clicking on the locations
highlight and go to those locations in the module.

\item[34. Feature (Stack trace)] \mbox{}\\
%
Make clicking on the action location highlight and go to that
location.
    
\item[35. Minor Bug (TLC Error report)] \mbox{}\\
%
Remove the @START...  and @END...  lines from the error message shown
to the user in the TLC Error window.  

\item[36. Feature (Action Fairness)] \mbox{}\\
%
Enhance the Init/Next method of writing the spec to add lists of
weakly and strongly fair actions.
    
\item[39. Minor Feature (Constant Assignment)] \mbox{}\\
%
When the user has assigned a set of model values to a constant
and specified symmetry, the default on the next page of the wizard
should be to make the values untyped.  

\item[40. Bug (Help)] \mbox{}\\
%
The help menu is not getting the focus if it doesn't already have the
focus and it either is in or is being opened into a folder that has
the focus.

% \item[41. Bug (View option)] \mbox{}\\
% %
% This option is not implemented.

\item[43. Minor Bug (Model Checking Results page display)] \mbox{}\\
%
The layout/spacing of the statistics / progress output / user output
fields sucks.  It might be better to make a two-column view like in
the other two pages, splitting the progress output into two separate
fields, one in each column.

\item[45. Minor Parsing Bug (Naming in nested ASSUME/PROVEs)] \mbox{}\\
%
Here is the example:
\begin{verbatim}
  THEOREM B == ASSUME TRUE, 
                      ASSUME NEW Z PROVE FALSE
               PROVE  FALSE
  <1>1. B!2!2  \* THIS IS A Bug, IT SHOULD NOT BE ALLOWED
  <1>2. QED
  Bar == B!2!2 \* THIS IS A Bug, IT SHOULD NOT BE ALLOWED
\end{verbatim}

\item[46. Bug (Model reports bogus ``TLC still running'')] \mbox{}\\
%
Sometimes when restarting the toolbox after it was closed with a model
open, the model says that TLC is still running.  The ``stop TLC'' button
in the upper right is red, but clicking it does nothing.  
The last time this happened, the Spec Explorer said that the model
has crashed, and clicking on repair solved the problem.
  
% \item[50. Minor Bug (Deleting a Spec)] \mbox{}\\
% %
% Deleting a spec throws a null pointer exception deep inside the call
% of createLink at ResourceHelper.java, line 301.  It would be a good
% idea to find out why, since this may be a symptom of an undiscovered
% bug.  (If it's benign, the exception should be caught and dealt with.)

\item[51. Very Minor Bug (Re-creating a spec)] \mbox{}\\
%
When creating a new spec, the Import Existing spec option is always
checked and disabled.  It should be unchecked and disabled if the
\texttt{.toolbox} directory does not already exist.  However, if the 
\texttt{.toolbox}
directory does exist, its present behavior of checking and enabling
the box is the correct one.  

\item[53. Minor Bug (Eclipse resource saving?)] \mbox{}\\
%
If I have automatic parse turned off, modify a file, and then try to
parse it, the Toolbox pops up a dialog saying I have a modified
resource and asking if I want to save it.  When I say yes, it closes
the dialog and goes ahead and parses the spec without saving the file.

\item[54. Minor Bug (Open Module)] \mbox{}\\
%
When the user clicks on File/Open Module/Choose \TLA+ Module, the file
browser that pops up should show the specification's directory.

\item[57. Bug (Pluscal Translator preferences)] \mbox{}\\
%
The PlusCal translator preferences seem to be no-ops.  The
``Re-translate on module save'' option should be made to work.  This
should cause the PlusCal translator to be called whenever a module is
saved that contains the string \texttt{--algorithm}.

The ``Popup problem window on translator errors'' is currently
redundant, since translator errors are shown in the Parsing
Errors view, which has its own preference for controlling that.
    
% \item[60. Minor Bug (Model Validation)] \mbox{}\\
% %
% When the model is validated and finds an error, the existence of that
% error should be reported at the top of all the pages---or at least at
% the top of the current page.  Now, the user can type something on one
% page that causes an error on another page and have no idea what's
% going on.

\item[62. Minor Bug (Distribution)] \mbox{}\\
%
	Clicking on Help on the page that appears after closing a
spec 	goes to a page not found screen in the windows distribution.

\item[63. Missing Feature (TLC Progress)] \mbox{}\\
%
The Model Checking Results page should list in the General section the
information about probability of fingerprint collision that TLC prints
after a run that finishes computing the reachable states.  (I think it
prints it out in the event of a liveness error.)  It should say
approximately something like:
  Probability of fingerprint collision:
     theoretical: 4.3E-15  pragmatic: 8.7E-15

\item[64. TLC Bug (Liveness checking in Simulation mode)] \mbox{}\\
%
TLC's liveness checking in simulation mode is seriously buggy.  
This is unlikely to be fixed soon, so we'll
have to tell users not to check liveness properties in simulation
mode.  

\item[66. Minor Missing Feature (Results page/Simulation mode)] \mbox{}\\
%
When TLC finds an error in simulation mode, it prints out a ``seed'' and
``aril'' that can be used to re-create the error.  It would be nice if
the Toolbox reported those values in the General section of the
results page.  This is not very important because the Trace Explorer
should make it unnecessary for the user to want to use them.

\item[67. Missing Feature (saved spec files)] \mbox{}\\
%
It should be possible for the user to open in a read-only editor
window the copies of the module files saved when a model is run.  The
menu raised by right-clicking on the model in the Spec Explorer should
have an ``open saved module'' option.  Clicking on it should raise a
file browser on the model's directory that shows all the tla files
there.  (It wouldn't hurt to allow the user to browse the MC.tla and
even the MC.cfg file as well.)

\item[68. Minor Missing Feature (MC file conflict)] \mbox{}\\
%
We should change the model validator to check if the spec contains a
module named MC and to pop up an error window if it does saying that
the spec can't be model checked.
  
% \item[69. Minor Bug (blank and empty expression fields)] \mbox{}\\
% %
% Entering space characters in various fields of the model editor that
% should have expressions creates errors that looks mysterious and can
% be hard to figure out.  For example, look at what happens if you type
% a blank Action Constraint and validate the model.  Or if you create an
% invariant and leave the field blank.  Typing just blanks in any field
% like Action Constraint should be equivalent to not typing anything.
% In something like an invariant or a definition override, a blank or
% empty entry should be treated as if the user canceled the operation.
% (The one place it doesn't matter is in the Additional Definitioins
% section.)

% \item[70. Bug (View option)] \mbox{}\\
% The View option on the Advanced Run section is being ignored.

\item[71. Bug/Missing Feature (MC file parsing error bubbles.)] \mbox{}\\
Putting the mouse cursor over a model error bubble produced by a 
parsing error in the MC file produces a message that mentions
the location in the MC file.  It needs to be translated to
a location in the model.

\item[72. Minor Bug (Advanced Model Values)] \mbox{}\\
The Toolbox reports an error if a space is left after a model
value typed into the Model Values section of the advanced
model editor page.

\item[73. Bug (Opening TLC error view on Linux)] \mbox{}\\
%
Opening the TLC error view on Linux causes the program to enter an
infinite loop.  

\end{description}
\end{document}

%try
%try