\batchmode %% Suppresses most terminal output.
\documentclass{article}
\usepackage{color}
\definecolor{boxshade}{gray}{0.85}
\setlength{\textwidth}{360pt}
\setlength{\textheight}{541pt}
\usepackage{latexsym}
\usepackage{ifthen}
% \usepackage{color}
%%%%%%%%%%%%%%%%%%%%%%%%%%%%%%%%%%%%%%%%%%%%%%%%%%%%%%%%%%%%%%%%%%%%%%%%%%%%%
% SWITCHES                                                                  %
%%%%%%%%%%%%%%%%%%%%%%%%%%%%%%%%%%%%%%%%%%%%%%%%%%%%%%%%%%%%%%%%%%%%%%%%%%%%%
\newboolean{shading} 
\setboolean{shading}{false}
\makeatletter
 %% this is needed only when inserted into the file, not when
 %% used as a package file.
%%%%%%%%%%%%%%%%%%%%%%%%%%%%%%%%%%%%%%%%%%%%%%%%%%%%%%%%%%%%%%%%%%%%%%%%%%%%%
%                                                                           %
% DEFINITIONS OF SYMBOL-PRODUCING COMMANDS                                  %
%                                                                           %
%    TLA+      LaTeX                                                        %
%    symbol    command                                                      %
%    ------    -------                                                      %
%    =>        \implies                                                     %
%    <:        \ltcolon                                                     %
%    :>        \colongt                                                     %
%    ==        \defeq                                                       %
%    ..        \dotdot                                                      %
%    ::        \coloncolon                                                  %
%    =|        \eqdash                                                      %
%    ++        \pp                                                          %
%    --        \mm                                                          %
%    **        \stst                                                        %
%    //        \slsl                                                        %
%    ^         \ct                                                          %
%    \A        \A                                                           %
%    \E        \E                                                           %
%    \AA       \AA                                                          %
%    \EE       \EE                                                          %
%%%%%%%%%%%%%%%%%%%%%%%%%%%%%%%%%%%%%%%%%%%%%%%%%%%%%%%%%%%%%%%%%%%%%%%%%%%%%
\newlength{\symlength}
\newcommand{\implies}{\Rightarrow}
\newcommand{\ltcolon}{\mathrel{<\!\!\mbox{:}}}
\newcommand{\colongt}{\mathrel{\!\mbox{:}\!\!>}}
\newcommand{\defeq}{\;\mathrel{\smash   %% keep this symbol from being too tall
    {{\stackrel{\scriptscriptstyle\Delta}{=}}}}\;}
\newcommand{\dotdot}{\mathrel{\ldotp\ldotp}}
\newcommand{\coloncolon}{\mathrel{::\;}}
\newcommand{\eqdash}{\mathrel = \joinrel \hspace{-.28em}|}
\newcommand{\pp}{\mathbin{++}}
\newcommand{\mm}{\mathbin{--}}
\newcommand{\stst}{*\!*}
\newcommand{\slsl}{/\!/}
\newcommand{\ct}{\hat{\hspace{.4em}}}
\newcommand{\A}{\forall}
\newcommand{\E}{\exists}
\renewcommand{\AA}{\makebox{$\raisebox{.05em}{\makebox[0pt][l]{%
   $\forall\hspace{-.517em}\forall\hspace{-.517em}\forall$}}%
   \forall\hspace{-.517em}\forall \hspace{-.517em}\forall\,$}}
\newcommand{\EE}{\makebox{$\raisebox{.05em}{\makebox[0pt][l]{%
   $\exists\hspace{-.517em}\exists\hspace{-.517em}\exists$}}%
   \exists\hspace{-.517em}\exists\hspace{-.517em}\exists\,$}}
\newcommand{\whileop}{\.{\stackrel
  {\mbox{\raisebox{-.3em}[0pt][0pt]{$\scriptscriptstyle+\;\,$}}}%
  {-\hspace{-.16em}\triangleright}}}

% Commands are defined to produce the upper-case keywords.
% Note that some have space after them.
\newcommand{\ASSUME}{\textsc{assume }}
\newcommand{\ASSUMPTION}{\textsc{assumption }}
\newcommand{\AXIOM}{\textsc{axiom }}
\newcommand{\BOOLEAN}{\textsc{boolean }}
\newcommand{\CASE}{\textsc{case }}
\newcommand{\CONSTANT}{\textsc{constant }}
\newcommand{\CONSTANTS}{\textsc{constants }}
\newcommand{\ELSE}{\settowidth{\symlength}{\THEN}%
   \makebox[\symlength][l]{\textsc{ else}}}
\newcommand{\EXCEPT}{\textsc{ except }}
\newcommand{\EXTENDS}{\textsc{extends }}
\newcommand{\FALSE}{\textsc{false}}
\newcommand{\IF}{\textsc{if }}
\newcommand{\IN}{\settowidth{\symlength}{\LET}%
   \makebox[\symlength][l]{\textsc{in}}}
\newcommand{\INSTANCE}{\textsc{instance }}
\newcommand{\LET}{\textsc{let }}
\newcommand{\LOCAL}{\textsc{local }}
\newcommand{\MODULE}{\textsc{module }}
\newcommand{\OTHER}{\textsc{other }}
\newcommand{\STRING}{\textsc{string}}
\newcommand{\THEN}{\textsc{ then }}
\newcommand{\THEOREM}{\textsc{theorem }}
\newcommand{\LEMMA}{\textsc{lemma }}
\newcommand{\PROPOSITION}{\textsc{proposition }}
\newcommand{\COROLLARY}{\textsc{corollary }}
\newcommand{\TRUE}{\textsc{true}}
\newcommand{\VARIABLE}{\textsc{variable }}
\newcommand{\VARIABLES}{\textsc{variables }}
\newcommand{\WITH}{\textsc{ with }}
\newcommand{\WF}{\textrm{WF}}
\newcommand{\SF}{\textrm{SF}}
\newcommand{\CHOOSE}{\textsc{choose }}
\newcommand{\ENABLED}{\textsc{enabled }}
\newcommand{\UNCHANGED}{\textsc{unchanged }}
\newcommand{\SUBSET}{\textsc{subset }}
\newcommand{\UNION}{\textsc{union }}
\newcommand{\DOMAIN}{\textsc{domain }}
% Added for tla2tex
\newcommand{\BY}{\textsc{by }}
\newcommand{\OBVIOUS}{\textsc{obvious }}
\newcommand{\HAVE}{\textsc{have }}
\newcommand{\QED}{\textsc{qed }}
\newcommand{\TAKE}{\textsc{take }}
\newcommand{\DEF}{\textsc{def }}
\newcommand{\HIDE}{\textsc{hide }}
\newcommand{\RECURSIVE}{\textsc{recursive }}
\newcommand{\USE}{\textsc{use }}
\newcommand{\DEFINE}{\textsc{define }}
\newcommand{\PROOF}{\textsc{proof }}
\newcommand{\WITNESS}{\textsc{witness }}
\newcommand{\PICK}{\textsc{pick }}
\newcommand{\DEFS}{\textsc{defs }}
\newcommand{\PROVE}{\settowidth{\symlength}{\ASSUME}%
   \makebox[\symlength][l]{\textsc{prove}}\@s{-4.1}}%
  %% The \@s{-4.1) is a kludge added on 24 Oct 2009 [happy birthday, Ellen]
  %% so the correct alignment occurs if the user types
  %%   ASSUME X
  %%   PROVE  Y
  %% because it cancels the extra 4.1 pts added because of the 
  %% extra space after the PROVE.  This seems to works OK.
  %% However, the 4.1 equals Parameters.LaTeXLeftSpace(1) and
  %% should be changed if that method ever changes.
\newcommand{\SUFFICES}{\textsc{suffices }}
\newcommand{\NEW}{\textsc{new }}
\newcommand{\LAMBDA}{\textsc{lambda }}
\newcommand{\STATE}{\textsc{state }}
\newcommand{\ACTION}{\textsc{action }}
\newcommand{\TEMPORAL}{\textsc{temporal }}
\newcommand{\ONLY}{\textsc{only }}              %% added by LL on 2 Oct 2009
\newcommand{\OMITTED}{\textsc{omitted }}        %% added by LL on 31 Oct 2009
\newcommand{\@pfstepnum}[2]{\ensuremath{\langle#1\rangle}\textrm{#2}}
\newcommand{\bang}{\@s{1}\mbox{\small !}\@s{1}}

%%%%%%%%%%%%%%%%%%%%%%%%%%%%%%%%%%%%%%%%%%%%%%%%%%%%%%%%%
% REDEFINE STANDARD COMMANDS TO MAKE THEM FORMAT BETTER %
%                                                       %
% We redefine \in and \notin                            %
%%%%%%%%%%%%%%%%%%%%%%%%%%%%%%%%%%%%%%%%%%%%%%%%%%%%%%%%%
\renewcommand{\_}{\rule{.4em}{.06em}\hspace{.05em}}
\newlength{\equalswidth}
\let\oldin=\in
\let\oldnotin=\notin
\renewcommand{\in}{%
   {\settowidth{\equalswidth}{$\.{=}$}\makebox[\equalswidth][c]{$\oldin$}}}
\renewcommand{\notin}{%
   {\settowidth{\equalswidth}{$\.{=}$}\makebox[\equalswidth]{$\oldnotin$}}}


%%%%%%%%%%%%%%%%%%%%%%%%%%%%%%%%%%%%%%%%%%%%%%%%%%%%
%                                                  %
% HORIZONTAL BARS:                                 %
%                                                  %
%   \moduleLeftDash    |~~~~~~~~~~                 %
%   \moduleRightDash    ~~~~~~~~~~|                %
%   \midbar            |----------|                %
%   \bottombar         |__________|                %
%%%%%%%%%%%%%%%%%%%%%%%%%%%%%%%%%%%%%%%%%%%%%%%%%%%%
\newlength{\charwidth}\settowidth{\charwidth}{{\small\tt M}}
\newlength{\boxrulewd}\setlength{\boxrulewd}{.4pt}
\newlength{\boxlineht}\setlength{\boxlineht}{.5\baselineskip}
\newcommand{\boxsep}{\charwidth}
\newlength{\boxruleht}\setlength{\boxruleht}{.5ex}
\newlength{\boxruledp}\setlength{\boxruledp}{-\boxruleht}
\addtolength{\boxruledp}{\boxrulewd}
\newcommand{\boxrule}{\leaders\hrule height \boxruleht depth \boxruledp
                      \hfill\mbox{}}
\newcommand{\@computerule}{%
  \setlength{\boxruleht}{.5ex}%
  \setlength{\boxruledp}{-\boxruleht}%
  \addtolength{\boxruledp}{\boxrulewd}}

\newcommand{\bottombar}{\hspace{-\boxsep}%
  \raisebox{-\boxrulewd}[0pt][0pt]{\rule[.5ex]{\boxrulewd}{\boxlineht}}%
  \boxrule
  \raisebox{-\boxrulewd}[0pt][0pt]{%
      \rule[.5ex]{\boxrulewd}{\boxlineht}}\hspace{-\boxsep}\vspace{0pt}}

\newcommand{\moduleLeftDash}%
   {\hspace*{-\boxsep}%
     \raisebox{-\boxlineht}[0pt][0pt]{\rule[.5ex]{\boxrulewd
               }{\boxlineht}}%
    \boxrule\hspace*{.4em }}

\newcommand{\moduleRightDash}%
    {\hspace*{.4em}\boxrule
    \raisebox{-\boxlineht}[0pt][0pt]{\rule[.5ex]{\boxrulewd
               }{\boxlineht}}\hspace{-\boxsep}}%\vspace{.2em}

\newcommand{\midbar}{\hspace{-\boxsep}\raisebox{-.5\boxlineht}[0pt][0pt]{%
   \rule[.5ex]{\boxrulewd}{\boxlineht}}\boxrule\raisebox{-.5\boxlineht%
   }[0pt][0pt]{\rule[.5ex]{\boxrulewd}{\boxlineht}}\hspace{-\boxsep}}

%%%%%%%%%%%%%%%%%%%%%%%%%%%%%%%%%%%%%%%%%%%%%%%%%%%%%%%%%%%%%%%%%%%%%%%%%%%%%
% FORMATING COMMANDS                                                        %
%%%%%%%%%%%%%%%%%%%%%%%%%%%%%%%%%%%%%%%%%%%%%%%%%%%%%%%%%%%%%%%%%%%%%%%%%%%%%


%\tstrut: A strut to produce inter-paragraph space in a comment.
%\rstrut: A strut to extend the bottom of a one-line comment so
%         there's no break in the shading between comments on 
%         successive lines.
\newcommand\tstrut%
  {\raisebox{\vshadelen}{\raisebox{-.25em}{\rule{0pt}{1.15em}}}%
   \global\setlength{\vshadelen}{0pt}}
\newcommand\rstrut{\raisebox{-.25em}{\rule{0pt}{1.15em}}%
 \global\setlength{\vshadelen}{0pt}}


% \.{op} formats operator op in math mode with empty boxes on either side.
% Used because TeX otherwise vary the amount of space it leaves around op.
\renewcommand{\.}[1]{\ensuremath{\mbox{}#1\mbox{}}}

% \@s{n} produces an n-point space
\newcommand{\@s}[1]{\hspace{#1pt}}           

% \@x{txt} starts a specification line in the beginning with txt
% in the final LaTeX source.
\newcommand{\@x}[1]{\par\mbox{$\mbox{}#1\mbox{}$}}  

% \@xx{txt} continues a specification line with the text txt.
\newcommand{\@xx}[1]{\mbox{$\mbox{}#1\mbox{}$}}  

% \@y{cmt} produces a one-line comment.
\newcommand{\@y}[1]{\mbox{\footnotesize\hspace{.65em}%
  \ifthenelse{\boolean{shading}}{%
      \shadebox{#1\hspace{-\the\lastskip}\rstrut}}%
               {#1\hspace{-\the\lastskip}\rstrut}}}

% \@z{cmt} produces a zero-width one-line comment.
\newcommand{\@z}[1]{\makebox[0pt][l]{\footnotesize
  \ifthenelse{\boolean{shading}}{%
      \shadebox{#1\hspace{-\the\lastskip}\rstrut}}%
               {#1\hspace{-\the\lastskip}\rstrut}}}


% \@w{str} produces the TLA+ string "str".
\newcommand{\@w}[1]{\textsf{``{#1}''}}             


%%%%%%%%%%%%%%%%%%%%%%%%%%%%%%%%%%%%%%%%%%%%%%%%%%%%%%%%%%%%%%%%%%%%%%%%%%%%%
% SHADING                                                                   %
%%%%%%%%%%%%%%%%%%%%%%%%%%%%%%%%%%%%%%%%%%%%%%%%%%%%%%%%%%%%%%%%%%%%%%%%%%%%%
\def\graymargin{1}
  % The number of points of margin in the shaded box.

% \definecolor{boxshade}{gray}{.85}
% Defines the darkness of the shading: 1 = white, 0 = black
% Added by TLATeX only if needed.

% \shadebox{txt} puts txt in a shaded box.
\newlength{\templena}
\newlength{\templenb}
\newsavebox{\tempboxa}
\newcommand{\shadebox}[1]{{\setlength{\fboxsep}{\graymargin pt}%
     \savebox{\tempboxa}{#1}%
     \settoheight{\templena}{\usebox{\tempboxa}}%
     \settodepth{\templenb}{\usebox{\tempboxa}}%
     \hspace*{-\fboxsep}\raisebox{0pt}[\templena][\templenb]%
        {\colorbox{boxshade}{\usebox{\tempboxa}}}\hspace*{-\fboxsep}}}

% \vshade{n} makes an n-point inter-paragraph space, with
%  shading if the `shading' flag is true.
\newlength{\vshadelen}
\setlength{\vshadelen}{0pt}
\newcommand{\vshade}[1]{\ifthenelse{\boolean{shading}}%
   {\global\setlength{\vshadelen}{#1pt}}%
   {\vspace{#1pt}}}

\newlength{\boxwidth}
\newlength{\multicommentdepth}

%%%%%%%%%%%%%%%%%%%%%%%%%%%%%%%%%%%%%%%%%%%%%%%%%%%%%%%%%%%%%%%%%%%%%%%%%%%%%
% THE cpar ENVIRONMENT                                                      %
% ^^^^^^^^^^^^^^^^^^^^                                                      %
% The LaTeX input                                                           %
%                                                                           %
%   \begin{cpar}{pop}{nest}{isLabel}{d}{e}{arg6}                            %
%     XXXXXXXXXXXXXXX                                                       %
%     XXXXXXXXXXXXXXX                                                       %
%     XXXXXXXXXXXXXXX                                                       %
%   \end{cpar}                                                              %
%                                                                           %
% produces one of two possible results.  If isLabel is the letter "T",      %
% it produces the following, where [label] is the result of typesetting     %
% arg6 in an LR box, and d is is a number representing a distance in        %
% points.                                                                   %
%                                                                           %
%   prevailing |<-- d -->[label]<- e ->XXXXXXXXXXXXXXX                      %
%         left |                       XXXXXXXXXXXXXXX                      %
%       margin |                       XXXXXXXXXXXXXXX                      %
%                                                                           %
% If isLabel is the letter "F", then it produces                            %
%                                                                           %
%   prevailing |<-- d -->XXXXXXXXXXXXXXXXXXXXXXX                            %
%         left |         <- e ->XXXXXXXXXXXXXXXX                            %
%       margin |                XXXXXXXXXXXXXXXX                            %
%                                                                           %
% where d and e are numbers representing distances in points.               %
%                                                                           %
% The prevailing left margin is the one in effect before the most recent    %
% pop (argument 1) cpar environments with "T" as the nest argument, where   %
% pop is a number \geq 0.                                                   %
%                                                                           %
% If the nest argument is the letter "T", then the prevailing left          %
% margin is moved to the left of the second (and following) lines of        %
% X's.  Otherwise, the prevailing left margin is left unchanged.            %
%                                                                           %
% An \unnest{n} command moves the prevailing left margin to where it was    %
% before the most recent n cpar environments with "T" as the nesting        %
% argument.                                                                 %
%                                                                           %
% The environment leaves no vertical space above or below it, or between    %
% its paragraphs.  (TLATeX inserts the proper amount of vertical space.)    %
%%%%%%%%%%%%%%%%%%%%%%%%%%%%%%%%%%%%%%%%%%%%%%%%%%%%%%%%%%%%%%%%%%%%%%%%%%%%%

\newcounter{pardepth}
\setcounter{pardepth}{0}

% \setgmargin{txt} defines \gmarginN to be txt, where N is \roman{pardepth}.
% \thegmargin equals \gmarginN, where N is \roman{pardepth}.
\newcommand{\setgmargin}[1]{%
  \expandafter\xdef\csname gmargin\roman{pardepth}\endcsname{#1}}
\newcommand{\thegmargin}{\csname gmargin\roman{pardepth}\endcsname}
\newcommand{\gmargin}{0pt}

\newsavebox{\tempsbox}

\newenvironment{cpar}[6]{%
  \addtocounter{pardepth}{-#1}%
  \ifthenelse{\boolean{shading}}{\par\begin{lrbox}{\tempsbox}%
                                 \begin{minipage}[t]{\linewidth}}{}%
  \begin{list}{}{%
     \edef\temp{\thegmargin}
     \ifthenelse{\equal{#3}{T}}%
       {\settowidth{\leftmargin}{\hspace{\temp}\footnotesize #6\hspace{#5pt}}%
        \addtolength{\leftmargin}{#4pt}}%
       {\setlength{\leftmargin}{#4pt}%
        \addtolength{\leftmargin}{#5pt}%
        \addtolength{\leftmargin}{\temp}%
        \setlength{\itemindent}{-#5pt}}%
      \ifthenelse{\equal{#2}{T}}{\addtocounter{pardepth}{1}%
                                 \setgmargin{\the\leftmargin}}{}%
      \setlength{\labelwidth}{0pt}%
      \setlength{\labelsep}{0pt}%
      \setlength{\itemindent}{-\leftmargin}%
      \setlength{\topsep}{0pt}%
      \setlength{\parsep}{0pt}%
      \setlength{\partopsep}{0pt}%
      \setlength{\parskip}{0pt}%
      \setlength{\itemsep}{0pt}
      \setlength{\itemindent}{#4pt}%
      \addtolength{\itemindent}{-\leftmargin}}%
   \ifthenelse{\equal{#3}{T}}%
      {\item[\tstrut\footnotesize \hspace{\temp}{#6}\hspace{#5pt}]
        }%
      {\item[\tstrut\hspace{\temp}]%
         }%
   \footnotesize}
 {\hspace{-\the\lastskip}\tstrut
 \end{list}%
  \ifthenelse{\boolean{shading}}{\end{minipage}  
                                 \end{lrbox}%
                                 \shadebox{\usebox{\tempsbox}}\par}{}%
  }

%%%%%%%%%%%%%%%%%%%%%%%%%%%%%%%%%%%%%%%%%%%%%%%%%%%%%%%%%%%%%%%%%%%%%%%%%%%%%%
% THE lcom ENVIRONMENT                                                       %
% ^^^^^^^^^^^^^^^^^^^^                                                       %
% A multi-line comment with no text to its left is typeset in an lcom        % 
% environment, whose argument is a number representing the indentation       % 
% of the left margin, in points.  All the text of the comment should be      % 
% inside cpar environments.                                                  % 
%%%%%%%%%%%%%%%%%%%%%%%%%%%%%%%%%%%%%%%%%%%%%%%%%%%%%%%%%%%%%%%%%%%%%%%%%%%%%%
\newenvironment{lcom}[1]{%
  \par\vspace{.2em}%
  \sloppypar
  \setcounter{pardepth}{0}%
  \footnotesize
  \begin{list}{}{%
    \setlength{\leftmargin}{#1pt}
    \setlength{\labelwidth}{0pt}%
    \setlength{\labelsep}{0pt}%
    \setlength{\itemindent}{0pt}%
    \setlength{\topsep}{0pt}%
    \setlength{\parsep}{0pt}%
    \setlength{\partopsep}{0pt}%
    \setlength{\parskip}{0pt}}
    \item[]}%
  {\end{list}\vspace{.3em}%
 }


%%%%%%%%%%%%%%%%%%%%%%%%%%%%%%%%%%%%%%%%%%%%%%%%%%%%%%%%%%%%%%%%%%%%%%%%%%%%%
% THE mcom ENVIRONMENT AND \mutivspace COMMAND                              %
% ^^^^^^^^^^^^^^^^^^^^^^^^^^^^^^^^^^^^^^^^^^^^                              %
%                                                                           %
% A part of the spec containing a right-comment of the form                 %
%                                                                           %
%      xxxx (*************)                                                 %
%      yyyy (* ccccccccc *)                                                 %
%      ...  (* ccccccccc *)                                                 %
%           (* ccccccccc *)                                                 %
%           (* ccccccccc *)                                                 %
%           (*************)                                                 %
%                                                                           %
% is typeset by                                                             %
%                                                                           %
%     XXXX \begin{mcom}{d}                                                  %
%            CCCC ... CCC                                                   %
%          \end{mcom}                                                       %
%     YYYY ...                                                              %
%     \multivspace{n}                                                       %
%                                                                           %
% where the number d is the width in points of the comment, n is the        %
% number of xxxx, yyyy, ...  lines to the left of the comment.              %
% All the text of the comment should be typeset in cpar environments.       %
%                                                                           %
% This puts the comment into a single box (so no page breaks can occur      %
% within it).  The entire box is shaded iff the shading flag is true.       %
%%%%%%%%%%%%%%%%%%%%%%%%%%%%%%%%%%%%%%%%%%%%%%%%%%%%%%%%%%%%%%%%%%%%%%%%%%%%%
\newenvironment{mcom}[1]{%
  \setcounter{pardepth}{0}%
  \hspace{.65em}%
  \begin{lrbox}{\alignbox}\sloppypar%
      \setboolean{shading}{false}%
      \setlength{\boxwidth}{#1pt}%
      \addtolength{\boxwidth}{-.65em}%
      \begin{minipage}[t]{\boxwidth}\footnotesize
      \parskip=0pt\relax}%
       {\end{minipage}\end{lrbox}%
       \settodepth{\alignwidth}{\usebox{\alignbox}}%
      \global\setlength{\multicommentdepth}{\alignwidth}%
      \global\addtolength{\alignwidth}{-\maxdepth}%
      \raisebox{0pt}[0pt][0pt]{%
        \ifthenelse{\boolean{shading}}%
          {\shadebox{\usebox{\alignbox}}}%
          {\usebox{\alignbox}}}%
       \vspace*{\alignwidth}\pagebreak[0]\vspace{-\alignwidth}\par}
 % a multi-line comment, whose first argument is its width in points.


% \multispace{n} produces the vertical space indicated by "|"s in 
% this situation
%   
%     xxxx (*************)
%     xxxx (* ccccccccc *)
%      |   (* ccccccccc *)
%      |   (* ccccccccc *)
%      |   (* ccccccccc *)
%      |   (*************)
%
% where n is the number of "xxxx" lines.
\newcommand{\multivspace}[1]{\addtolength{\multicommentdepth}{-#1\baselineskip}%
 \addtolength{\multicommentdepth}{1.2em}%
 \ifthenelse{\lengthtest{\multicommentdepth > 0pt}}%
    {\par\vspace{\multicommentdepth}\par}{}}

%\newenvironment{hpar}[2]{%
%  \begin{list}{}{\setlength{\leftmargin}{#1pt}%
%                 \addtolength{\leftmargin}{#2pt}%
%                 \setlength{\itemindent}{-#2pt}%
%                 \setlength{\topsep}{0pt}%
%                 \setlength{\parsep}{0pt}%
%                 \setlength{\partopsep}{0pt}%
%                 \setlength{\parskip}{0pt}%
%                 \addtolength{\labelsep}{0pt}}%
%  \item[]\footnotesize}{\end{list}}
%    %%%%%%%%%%%%%%%%%%%%%%%%%%%%%%%%%%%%%%%%%%%%%%%%%%%%%%%%%%%%%%%%%%%%%%%%
%    % Typesets a sequence of paragraphs like this:                         %
%    %                                                                      %
%    %      left |<-- d1 --> XXXXXXXXXXXXXXXXXXXXXXXX                       %
%    %    margin |           <- d2 -> XXXXXXXXXXXXXXX                       %
%    %           |                    XXXXXXXXXXXXXXX                       %
%    %           |                                                          %
%    %           |                    XXXXXXXXXXXXXXX                       %
%    %           |                    XXXXXXXXXXXXXXX                       %
%    %                                                                      %
%    % where d1 = #1pt and d2 = #2pt, but with no vspace between            %
%    % paragraphs.                                                          %
%    %%%%%%%%%%%%%%%%%%%%%%%%%%%%%%%%%%%%%%%%%%%%%%%%%%%%%%%%%%%%%%%%%%%%%%%%

%%%%%%%%%%%%%%%%%%%%%%%%%%%%%%%%%%%%%%%%%%%%%%%%%%%%%%%%%%%%%%%%%%%%%%
% Commands for repeated characters that produce dashes.              %
%%%%%%%%%%%%%%%%%%%%%%%%%%%%%%%%%%%%%%%%%%%%%%%%%%%%%%%%%%%%%%%%%%%%%%
% \raisedDash{wd}{ht}{thk} makes a horizontal line wd characters wide, 
% raised a distance ht ex's above the baseline, with a thickness of 
% thk em's.
\newcommand{\raisedDash}[3]{\raisebox{#2ex}{\setlength{\alignwidth}{.5em}%
  \rule{#1\alignwidth}{#3em}}}

% The following commands take a single argument n and produce the
% output for n repeated characters, as follows
%   \cdash:    -
%   \tdash:    ~
%   \ceqdash:  =
%   \usdash:   _
\newcommand{\cdash}[1]{\raisedDash{#1}{.5}{.04}}
\newcommand{\usdash}[1]{\raisedDash{#1}{0}{.04}}
\newcommand{\ceqdash}[1]{\raisedDash{#1}{.5}{.08}}
\newcommand{\tdash}[1]{\raisedDash{#1}{1}{.08}}

\newlength{\spacewidth}
\setlength{\spacewidth}{.2em}
\newcommand{\e}[1]{\hspace{#1\spacewidth}}
%% \e{i} produces space corresponding to i input spaces.


%% Alignment-file Commands

\newlength{\alignboxwidth}
\newlength{\alignwidth}
\newsavebox{\alignbox}

% \al{i}{j}{txt} is used in the alignment file to put "%{i}{j}{wd}"
% in the log file, where wd is the width of the line up to that point,
% and txt is the following text.
\newcommand{\al}[3]{%
  \typeout{\%{#1}{#2}{\the\alignwidth}}%
  \cl{#3}}

%% \cl{txt} continues a specification line in the alignment file
%% with text txt.
\newcommand{\cl}[1]{%
  \savebox{\alignbox}{\mbox{$\mbox{}#1\mbox{}$}}%
  \settowidth{\alignboxwidth}{\usebox{\alignbox}}%
  \addtolength{\alignwidth}{\alignboxwidth}%
  \usebox{\alignbox}}

% \fl{txt} in the alignment file begins a specification line that
% starts with the text txt.
\newcommand{\fl}[1]{%
  \par
  \savebox{\alignbox}{\mbox{$\mbox{}#1\mbox{}$}}%
  \settowidth{\alignwidth}{\usebox{\alignbox}}%
  \usebox{\alignbox}}



  
%%%%%%%%%%%%%%%%%%%%%%%%%%%%%%%%%%%%%%%%%%%%%%%%%%%%%%%%%%%%%%%%%%%%%%%%%%%%%
% Ordinarily, TeX typesets letters in math mode in a special math italic    %
% font.  This makes it typeset "it" to look like the product of the         %
% variables i and t, rather than like the word "it".  The following         %
% commands tell TeX to use an ordinary italic font instead.                 %
%%%%%%%%%%%%%%%%%%%%%%%%%%%%%%%%%%%%%%%%%%%%%%%%%%%%%%%%%%%%%%%%%%%%%%%%%%%%%
\ifx\documentclass\undefined
\else
  \DeclareSymbolFont{tlaitalics}{\encodingdefault}{cmr}{m}{it}
  \let\itfam\symtlaitalics
\fi

\makeatletter
\newcommand{\tlx@c}{\c@tlx@ctr\advance\c@tlx@ctr\@ne}
\newcounter{tlx@ctr}
\c@tlx@ctr=\itfam \multiply\c@tlx@ctr"100\relax \advance\c@tlx@ctr "7061\relax
\mathcode`a=\tlx@c \mathcode`b=\tlx@c \mathcode`c=\tlx@c \mathcode`d=\tlx@c
\mathcode`e=\tlx@c \mathcode`f=\tlx@c \mathcode`g=\tlx@c \mathcode`h=\tlx@c
\mathcode`i=\tlx@c \mathcode`j=\tlx@c \mathcode`k=\tlx@c \mathcode`l=\tlx@c
\mathcode`m=\tlx@c \mathcode`n=\tlx@c \mathcode`o=\tlx@c \mathcode`p=\tlx@c
\mathcode`q=\tlx@c \mathcode`r=\tlx@c \mathcode`s=\tlx@c \mathcode`t=\tlx@c
\mathcode`u=\tlx@c \mathcode`v=\tlx@c \mathcode`w=\tlx@c \mathcode`x=\tlx@c
\mathcode`y=\tlx@c \mathcode`z=\tlx@c
\c@tlx@ctr=\itfam \multiply\c@tlx@ctr"100\relax \advance\c@tlx@ctr "7041\relax
\mathcode`A=\tlx@c \mathcode`B=\tlx@c \mathcode`C=\tlx@c \mathcode`D=\tlx@c
\mathcode`E=\tlx@c \mathcode`F=\tlx@c \mathcode`G=\tlx@c \mathcode`H=\tlx@c
\mathcode`I=\tlx@c \mathcode`J=\tlx@c \mathcode`K=\tlx@c \mathcode`L=\tlx@c
\mathcode`M=\tlx@c \mathcode`N=\tlx@c \mathcode`O=\tlx@c \mathcode`P=\tlx@c
\mathcode`Q=\tlx@c \mathcode`R=\tlx@c \mathcode`S=\tlx@c \mathcode`T=\tlx@c
\mathcode`U=\tlx@c \mathcode`V=\tlx@c \mathcode`W=\tlx@c \mathcode`X=\tlx@c
\mathcode`Y=\tlx@c \mathcode`Z=\tlx@c
\makeatother

%%%%%%%%%%%%%%%%%%%%%%%%%%%%%%%%%%%%%%%%%%%%%%%%%%%%%%%%%%
%                THE describe ENVIRONMENT                %
%%%%%%%%%%%%%%%%%%%%%%%%%%%%%%%%%%%%%%%%%%%%%%%%%%%%%%%%%%
%
%
% It is like the description environment except it takes an argument
% ARG that should be the text of the widest label.  It adjusts the
% indentation so each item with label LABEL produces
%%      LABEL             blah blah blah
%%      <- width of ARG ->blah blah blah
%%                        blah blah blah
\newenvironment{describe}[1]%
   {\begin{list}{}{\settowidth{\labelwidth}{#1}%
            \setlength{\labelsep}{.5em}%
            \setlength{\leftmargin}{\labelwidth}% 
            \addtolength{\leftmargin}{\labelsep}%
            \addtolength{\leftmargin}{\parindent}%
            \def\makelabel##1{\rm ##1\hfill}}%
            \setlength{\topsep}{0pt}}%% 
                % Sets \topsep to 0 to reduce vertical space above
                % and below embedded displayed equations
   {\end{list}}

%   For tlatex.TeX
\usepackage{verbatim}
\makeatletter
\def\tla{\let\%\relax%
         \@bsphack
         \typeout{\%{\the\linewidth}}%
             \let\do\@makeother\dospecials\catcode`\^^M\active
             \let\verbatim@startline\relax
             \let\verbatim@addtoline\@gobble
             \let\verbatim@processline\relax
             \let\verbatim@finish\relax
             \verbatim@}
\let\endtla=\@esphack


% The tlatex environment is used by TLATeX.TeX to typeset TLA+.
% TLATeX.TLA starts its files by writing a \tlatex command.  This
% command/environment sets \parindent to 0 and defines \% to its
% standard definition because the writing of the log files is messed up
% if \% is defined to be something else.  It also executes
% \@computerule to determine the dimensions for the TLA horizonatl
% bars.
\newenvironment{tlatex}{\@computerule%
                        \setlength{\parindent}{0pt}%
                       \makeatletter\chardef\%=`\%}{}


% The notla environment produces no output.  You can turn a 
% tla environment to a notla environment to prevent tlatex.TeX from
% re-formatting the environment.

\def\notla{\let\%\relax%
         \@bsphack
             \let\do\@makeother\dospecials\catcode`\^^M\active
             \let\verbatim@startline\relax
             \let\verbatim@addtoline\@gobble
             \let\verbatim@processline\relax
             \let\verbatim@finish\relax
             \verbatim@}
\let\endnotla=\@esphack


%%%%%%%%%%%%%%%%%%%%%%%% end of tlatex.sty file %%%%%%%%%%%%%%%%%%%%%%% 
% last modified on Fri 22 October 2010 at  7:52:11 PST by lamport

\begin{document}
\tlatex
\setboolean{shading}{true}
\@x{}\moduleLeftDash\@xx{ {\MODULE} Promises}\moduleRightDash\@xx{}%
\@x{ {\EXTENDS} Integers ,\, Sequences ,\, FiniteSets}%
\par\vspace{8.0pt}%
\begin{lcom}{0}%
\begin{cpar}{0}{F}{F}{0}{0}{}%
Objects, Heaps, and Methods
\end{cpar}%
\end{lcom}%
\@x{ {\CONSTANT} Id ,\, Value}%
\begin{lcom}{8.2}%
\begin{cpar}{0}{F}{F}{0}{0}{}%
 A value represents something like an int or a boolean. An \ensuremath{Id} is
 the
 heap address of an object. We can represent \ensuremath{null} as some
 particular \ensuremath{Id}. We assume that an \ensuremath{Id} is not a
 \ensuremath{Value}.
\end{cpar}%
\end{lcom}%
\par\vspace{8.0pt}%
\@x{ {\ASSUME} Id \.{\cap} Value \.{=} \{ \}}%
\par\vspace{8.0pt}%
\begin{lcom}{0}%
\begin{cpar}{0}{T}{F}{60.0}{0}{}%
Some TLA+ Notation
\end{cpar}%
\vshade{5.0}%
\begin{cpar}{1}{F}{F}{0}{0}{}%
 A function \ensuremath{f} has a domain written \ensuremath{{\DOMAIN} f} . The
 function assigns a
 value \ensuremath{f[x]} for every element \ensuremath{x} in
 \ensuremath{{\DOMAIN} f}. The domain of a function
 can be an infinite set.
\end{cpar}%
\vshade{5.0}%
\begin{cpar}{0}{F}{F}{0}{0}{}%
 [\ensuremath{S \.{\rightarrow} T}] is the set of all functions \ensuremath{f}
 with domain \ensuremath{S} such that \ensuremath{f[x]} is
 in the set \ensuremath{T} for all \ensuremath{x} in \ensuremath{S}.
\end{cpar}%
\vshade{5.0}%
\begin{cpar}{0}{F}{F}{0}{0}{}%
 A record \ensuremath{r} is a function whose domain is a non-empty finite set
 of
 strings, where we can write \ensuremath{r.fldName} as an abbreviation for
 \ensuremath{r[\mbox{``}fldName\mbox{''}]}.
\end{cpar}%
\vshade{5.0}%
\begin{cpar}{0}{F}{F}{0}{0}{}%
 [\ensuremath{foo \.{\mapsto} 42}, bar \ensuremath{\.{\mapsto} v}] is the
 record \ensuremath{r} whose domain is \ensuremath{\{\mbox{``}foo\mbox{''},\,
 \mbox{``}bar\mbox{''}\}
} such that \ensuremath{r.foo \.{=} 42} and \ensuremath{r.bar \.{=} v}.
\end{cpar}%
\vshade{5.0}%
\begin{cpar}{0}{F}{F}{0}{0}{}%
 [\ensuremath{foo} : \ensuremath{Nat}, bar : \ensuremath{V}] is the set of all
 records [\ensuremath{foo \.{\mapsto} n}, bar \ensuremath{\.{\mapsto} v}]
 such that \ensuremath{n} is in \ensuremath{Nat} and \ensuremath{v} is in
 \ensuremath{V}.
\end{cpar}%
\end{lcom}%
\par\vspace{8.0pt}%
\@x{ Object \.{\defeq}}%
\begin{lcom}{8.2}%
\begin{cpar}{0}{F}{F}{0}{0}{}%
 An \ensuremath{Object} is a record containing a \ensuremath{type} field that
 is a string.
 The fields represent the fields of the object. I don\mbox{'}t bother
 representing usual types and classes. The classic class structure
 can be represented by an \ensuremath{Object} having a \ensuremath{class}
 field whose value is
 the \ensuremath{id} of an \ensuremath{Object} of type ``class''. The fields
 of the latter \ensuremath{Object
} would represent the static fields of the class. The representation
 of methods is described below.
\end{cpar}%
\vshade{5.0}%
\begin{cpar}{0}{F}{F}{0}{0}{}%
We assume that a \ensuremath{Value} is not an \ensuremath{Object}.
\end{cpar}%
\end{lcom}%
 \@x{\@s{8.2} \.{\LET} Rcd ( Labels ) \.{\defeq} \{ R \.{\in} [ Labels
 \.{\rightarrow} Value \.{\cup} Id ] \.{:} R . type \.{\in} {\STRING} \}}%
 \@x{\@s{28.59} LabelSets \.{\defeq} \{ S \.{\in} {\SUBSET} {\STRING} \.{:}
 IsFiniteSet ( S ) \.{\land}\@w{type} \.{\in} S \}}%
 \@x{\@s{8.2} \.{\IN}\@s{4.1} {\UNION} \{ Rcd ( Labels ) \.{:} Labels \.{\in}
 LabelSets \}}%
\par\vspace{8.0pt}%
\@x{ {\ASSUME} Object \.{\cap} Value \.{=} \{ \}}%
\par\vspace{8.0pt}%
 \@x{ Heap \.{\defeq} [ Id \.{\rightarrow} Object \.{\cup} \{ {\langle}
 {\rangle} \} ]}%
\begin{lcom}{8.2}%
\begin{cpar}{0}{F}{F}{0}{0}{}%
 A \ensuremath{Heap} maps an \ensuremath{Dd} either to an \ensuremath{Object}
 or to \ensuremath{{\langle} {\rangle}}, the latter meaning
 that the \ensuremath{Id} is not the \ensuremath{Id} of any object.
\end{cpar}%
\end{lcom}%
\par\vspace{8.0pt}%
\@x{ ReachableFrom ( obj ,\, H ) \.{\defeq}}%
\begin{lcom}{8.2}%
\begin{cpar}{0}{F}{F}{0}{0}{}%
 This is the set of \ensuremath{Ids} reachable from an \ensuremath{Object}
 \ensuremath{obj} in the heap \ensuremath{H}.
 If \ensuremath{obj} is not an \ensuremath{Object}, it is the empty set.
 Otherwise, it consists
 of all the \ensuremath{Ids} that are values of fields of \ensuremath{obj},
 together with all
 the \ensuremath{Ids} reachable from objects in heap \ensuremath{H} with
 those \ensuremath{Ids}. It is
 defined in terms of two operators \ensuremath{IdsOf} and \ensuremath{R}.
\end{cpar}%
\end{lcom}%
\@x{\@s{8.2} \.{\LET} IdsOf ( o ) \.{\defeq}}%
\begin{lcom}{36.79}%
\begin{cpar}{0}{F}{F}{0}{0}{}%
 The set of all values of fields of object \ensuremath{obj} that are
 \ensuremath{Ids}, or
 the empty set if \ensuremath{obj} is not an object.
\end{cpar}%
\end{lcom}%
\@x{\@s{36.79} {\IF} o \.{\in} Object}%
 \@x{\@s{44.99} \.{\THEN} \{ i \.{\in} \{ o [ x ] \.{:} x \.{\in} {\DOMAIN} o
 \} \.{:} i \.{\in} Id \}}%
\@x{\@s{44.99} \.{\ELSE} \{ \}}%
\par\vspace{8.0pt}%
\@x{\@s{32.69} R [ n \.{\in} Nat ] \.{\defeq}}%
\begin{lcom}{43.15}%
\begin{cpar}{0}{F}{F}{0}{0}{}%
 Defines \ensuremath{R[n]} to be the set of \ensuremath{Ids} reachable from
 \ensuremath{id} by a path
 of length at most \ensuremath{n} in the heap \ensuremath{H}.
\end{cpar}%
\end{lcom}%
\@x{\@s{43.15} {\IF} n \.{=} 0 \.{\THEN} IdsOf ( obj )}%
 \@x{\@s{80.03} \.{\ELSE} R [ n \.{-} 1 ] \.{\cup} \{ IdsOf ( H [ i ] ) \.{:}
 i \.{\in} R [ n \.{-} 1 ] \}}%
\@x{\@s{8.2} \.{\IN}\@s{8.2} {\UNION} \{ R [ n ] \.{:} n \.{\in} Nat \}}%
\par\vspace{8.0pt}%
\@x{ {\CONSTANT} Method ,\, Eval ( \_ ,\, \_ ,\, \_ ,\, \_ )}%
\begin{lcom}{8.2}%
\begin{cpar}{0}{F}{F}{0}{0}{}%
We assume that there are only static methods. A method specifies the
 result of executing a method of some object. We are considering the
 execution of a method to be an atomic action. The result of
 executing a method \ensuremath{M} of an object \ensuremath{obj} with a list
 \ensuremath{args} of arguments
 \ensuremath{args} when the value of the heap is \ensuremath{H} is
 \ensuremath{Eval(M,\, obj,\, args,\, H)}, which
 is a record consisting of two fields:
\end{cpar}%
\vshade{5.0}%
\begin{cpar}{0}{T}{F}{5.0}{0}{}%
- A \ensuremath{result} field that equals the value returned by the method.
\end{cpar}%
\vshade{5.0}%
\begin{cpar}{0}{F}{F}{0}{0}{}%
- A \ensuremath{heap} field that is the heap after the execution.
\end{cpar}%
\vshade{5.0}%
\begin{cpar}{1}{F}{F}{0}{0}{}%
For convenience, we assume that a method is a value.
\end{cpar}%
\end{lcom}%
\@x{ {\ASSUME} Method \.{\subseteq} Value}%
\@x{}\midbar\@xx{}%
\begin{lcom}{0}%
\begin{cpar}{0}{F}{F}{0}{0}{}%
Promises
\end{cpar}%
\end{lcom}%
\@x{ ResolvedPromise \.{\defeq}}%
\@x{\@s{8.2} [ type\@s{16.76} \.{:} \{\@w{promise} \} ,\,}%
\@x{\@s{10.97} resolved \.{:} \{ {\TRUE} \} ,\,}%
\@x{\@s{10.97} value\@s{11.90} \.{:} Value \.{\cup} Id}%
\@x{\@s{8.2} ]}%
\par\vspace{8.0pt}%
\@x{ UnresolvedPromise \.{\defeq}}%
\begin{lcom}{8.2}%
\begin{cpar}{0}{F}{F}{0}{0}{}%
A promise is resolved by executing a method to compute its value.
 There are two kinds of promises: \ensuremath{when} promises and the other
 kind.
 A \ensuremath{when} promise is returned by executing a
 \ensuremath{whenFulfilled} such as
\end{cpar}%
\vshade{5.0}%
\begin{cpar}{0}{T}{F}{7.5}{0}{}%
\ensuremath{foo.whenFulfilled(arg \.{\implies} some} \ensuremath{code)
}%
\end{cpar}%
\vshade{5.0}%
\begin{cpar}{1}{F}{F}{0}{0}{}%
The promise is specified by an object with these fields:
\end{cpar}%
\vshade{5.0}%
\begin{cpar}{0}{T}{T}{5.0}{2.5}{%
-}%
A single-argument \ensuremath{Method}, which is dynamically created by
 executing the \ensuremath{whenFulfilled}. In the example, it is the
 \ensuremath{Method
} that would be written in the class containing the expression as
\end{cpar}%
\vshade{5.0}%
\begin{cpar}{0}{T}{F}{7.5}{0}{}%
\ensuremath{newMethod(arg) \{ some} \ensuremath{code \}
}%
\end{cpar}%
\vshade{5.0}%
\begin{cpar}{2}{T}{T}{5.0}{2.5}{%
-}%
 The \ensuremath{Id} of the object for which the \ensuremath{whenFulfilled}
 method was
 executed. In the example, references to \ensuremath{this} in
 \ensuremath{some} \ensuremath{code
} refer to the fields of this object.
\end{cpar}%
\vshade{5.0}%
\begin{cpar}{1}{T}{T}{5.0}{2.5}{%
-}%
A promise for method\mbox{'}s the single argument. In the example,
 it is a promise for \ensuremath{foo}.
\end{cpar}%
\vshade{5.0}%
\begin{cpar}{1}{F}{F}{0}{0}{}%
A non-\ensuremath{when} promise is one produced by executing something like
\end{cpar}%
\vshade{5.0}%
\begin{cpar}{0}{T}{F}{5.0}{0}{}%
\ensuremath{foo.Bar(arg1,\, \.{\dots} ,\, argN)
}%
\end{cpar}%
\vshade{5.0}%
\begin{cpar}{1}{F}{F}{0}{0}{}%
The promise is specified by an object with these fields:
\end{cpar}%
\vshade{5.0}%
\begin{cpar}{0}{T}{T}{5.0}{2.5}{%
-}%
A \ensuremath{Method} that represents the Bar method of the appropriate
 class.
\end{cpar}%
\vshade{5.0}%
\begin{cpar}{1}{T}{F}{5.0}{0}{}%
- A promise for an \ensuremath{Id} of the object \ensuremath{foo}.
\end{cpar}%
\vshade{5.0}%
\begin{cpar}{0}{F}{F}{0}{0}{}%
- The argument list.
\end{cpar}%
\vshade{5.0}%
\begin{cpar}{1}{F}{F}{0}{0}{}%
Because a promise is an \ensuremath{Object}, promises can appear just like any
 other object in the heap. They can also be used as arguments to
 methods.
\end{cpar}%
\end{lcom}%
\@x{\@s{8.2} [ type\@s{31.57} \.{:} \{\@w{promise} \} ,\,}%
\@x{\@s{10.97} resolved\@s{14.80} \.{:} \{ {\FALSE} \} ,\,}%
\@x{\@s{10.97} method\@s{17.74} \.{:} Method ,\,}%
\@x{\@s{10.97} isWhen\@s{16.45} \.{:} \{ {\TRUE} \} ,\,}%
\@x{\@s{10.97} objId\@s{26.92} \.{:} Id ,\,}%
\@x{\@s{10.97} argPromise \.{:} Id}%
\@x{\@s{8.2} ]}%
\par\vspace{8.0pt}%
\@x{\@s{8.2} \.{\cup}}%
\par\vspace{8.0pt}%
\@x{\@s{8.2} [ type\@s{39.89} \.{:} \{\@w{promise} \} ,\,}%
\@x{\@s{10.97} resolved\@s{23.13} \.{:} \{ {\FALSE} \} ,\,}%
\@x{\@s{10.97} method\@s{26.07} \.{:} Method ,\,}%
\@x{\@s{10.97} isWhen\@s{24.78} \.{:} \{ {\FALSE} \} ,\,}%
\@x{\@s{10.97} objIdPromise \.{:} Id ,\,}%
\@x{\@s{10.97} args\@s{39.70} \.{:} Seq ( Value \.{\cup} Id )}%
\@x{\@s{8.2} ]}%
\par\vspace{8.0pt}%
\@x{ Promise \.{\defeq} ResolvedPromise \.{\cup} UnresolvedPromise}%
\par\vspace{8.0pt}%
 \@x{ OKObject \.{\defeq} \{ o \.{\in} Object \.{:} o . type \.{=}\@w{promise}
 \.{\implies} o \.{\in} Promise \}}%
\begin{lcom}{8.2}%
\begin{cpar}{0}{F}{F}{0}{0}{}%
 The set \ensuremath{Object} minus those objects of type
 \ensuremath{\@w{promise}} that are not in
 the set \ensuremath{Promise}.
\end{cpar}%
\end{lcom}%
\par\vspace{8.0pt}%
\@x{ OKHeap \.{\defeq}}%
\begin{lcom}{8.2}%
\begin{cpar}{0}{F}{F}{0}{0}{}%
The set of Heaps such that:
\end{cpar}%
\vshade{5.0}%
\begin{cpar}{0}{T}{T}{5.0}{2.5}{%
-}%
There are no dangling pointers (fields of objects
 that are \ensuremath{Ids} that point to nothing).
\end{cpar}%
\vshade{5.0}%
\begin{cpar}{1}{T}{F}{5.0}{0}{}%
- Every \ensuremath{Id} in that should be the \ensuremath{Id} of a promise is.
\end{cpar}%
\vshade{5.0}%
\begin{cpar}{0}{F}{F}{0}{0}{}%
- There are no cycles of promises
\end{cpar}%
\end{lcom}%
\@x{\@s{8.2} \{ H \.{\in} Heap \.{:}}%
\@x{\@s{17.30} \A\, id \.{\in} Id \.{:}}%
\@x{\@s{24.52} \.{\LET} obj \.{\defeq} H [ id ]}%
 \@x{\@s{24.52} \.{\IN}\@s{4.1} ( obj \.{\neq} {\langle} {\rangle} )
 \.{\implies}}%
\@x{\@s{61.11} \.{\land} obj \.{\in} OKObject}%
\@x{\@s{61.11} \.{\land} \A\, fldName \.{\in} {\DOMAIN} obj \.{:}}%
 \@x{\@s{79.44} ( obj [ fldName ] \.{\in} Id ) \.{\implies} ( H [ obj [
 fldName ] ] \.{\neq} {\langle} {\rangle} )}%
 \@x{\@s{61.11} \.{\land} ( obj \.{\in} Promise ) \.{\land} ( {\lnot} obj .
 resolved ) \.{\implies}}%
 \@x{\@s{84.31} \.{\land} {\IF} obj . isWhen \.{\THEN} H [ obj . argPromise ]
 \.{\in} Promise}%
\@x{\@s{157.82} \.{\ELSE} H [ obj . objIdPromise ] \.{\in} Promise}%
\@x{\@s{84.31} \.{\land} \.{\LET} thePromise ( i ) \.{\defeq}}%
\@x{\@s{124.02} {\IF} H [ i ] . resolved}%
\@x{\@s{132.22} \.{\THEN} i}%
 \@x{\@s{132.22} \.{\ELSE} {\IF} H [ i ] . isWhen \.{\THEN} H [ i ] .
 argPromise}%
\@x{\@s{230.42} \.{\ELSE} H [ i ] . objIdPromise}%
\@x{\@s{115.82} R [ n \.{\in} Nat ] \.{\defeq}}%
\@x{\@s{130.38} {\IF} n \.{=} 0 \.{\THEN} id}%
\@x{\@s{167.25} \.{\ELSE} thePromise ( R [ n \.{-} 1 ] )}%
 \@x{\@s{95.42} \.{\IN}\@s{4.1} \A\, n \.{\in} Nat \.{\,\backslash\,} \{ 0 \}
 \.{:} R [ n ] \.{\neq} id\@s{69.7} \}}%
\par\vspace{8.0pt}%
\@x{ ReachableWithoutPromises ( obj ,\, H ) \.{\defeq}}%
\begin{lcom}{8.2}%
\begin{cpar}{0}{F}{F}{0}{0}{}%
Like \ensuremath{ReachableFrom}, except that it does not follow links inside
 objects of type \ensuremath{\@w{promise}}, which as described above represent
 promises.
\end{cpar}%
\end{lcom}%
\@x{\@s{8.2} \.{\LET} IdsOf ( o ) \.{\defeq}}%
\begin{lcom}{36.79}%
\begin{cpar}{0}{F}{F}{0}{0}{}%
 The set of all values of fields of object \ensuremath{obj} that are
 \ensuremath{Ids}, or
 the empty set if \ensuremath{obj} is not an object.
\end{cpar}%
\end{lcom}%
 \@x{\@s{36.79} {\IF} ( o \.{\in} Object ) \.{\land} ( o . type
 \.{\neq}\@w{promise} )}%
 \@x{\@s{44.99} \.{\THEN} \{ i \.{\in} \{ o [ x ] \.{:} x \.{\in} {\DOMAIN} o
 \} \.{:} i \.{\in} Id \}}%
\@x{\@s{44.99} \.{\ELSE} \{ \}}%
\@x{\@s{32.69} R [ n \.{\in} Nat ] \.{\defeq}}%
\begin{lcom}{43.15}%
\begin{cpar}{0}{F}{F}{0}{0}{}%
 Defines \ensuremath{R[n]} to be the set of \ensuremath{Ids} reachable from
 \ensuremath{id} by a path
 of length at most \ensuremath{n} in the heap \ensuremath{H}.
\end{cpar}%
\end{lcom}%
\@x{\@s{43.15} {\IF} n \.{=} 0 \.{\THEN} IdsOf ( obj )}%
 \@x{\@s{80.03} \.{\ELSE} R [ n \.{-} 1 ] \.{\cup} \{ IdsOf ( H [ i ] ) \.{:}
 i \.{\in} R [ n \.{-} 1 ] \}}%
\@x{\@s{8.2} \.{\IN}\@s{8.2} {\UNION} \{ R [ n ] \.{:} n \.{\in} Nat \}}%
\par\vspace{16.0pt}%
\@x{ {\ASSUME} \A\, M \.{\in} Method ,\,}%
\@x{\@s{45.46} obj \.{\in} OKObject ,\,}%
\@x{\@s{45.46} args \.{\in} Seq ( Value \.{\cup} Id ) \.{:}}%
\@x{\@s{46.44} \.{\LET} E ( H ) \.{\defeq} Eval ( M ,\, obj ,\, args ,\, H )}%
 \@x{\@s{66.84} UseableId ( H ) \.{\defeq} ReachableWithoutPromises ( obj ,\,
 H ) \.{\cup}}%
 \@x{\@s{153.26} {\UNION} \{ ReachableWithoutPromises ( args [ i ] ,\, H )
 \.{:}}%
\@x{\@s{198.01} i \.{\in} 1 \.{\dotdot} Len ( args ) \}}%
\@x{\@s{46.44} \.{\IN}\@s{4.1} \.{\land} \A\, H \.{\in} OKHeap \.{:}}%
 \@x{\@s{90.25} \.{\land} E ( H ) \.{\in} [ result \.{:} Value \.{\cup} Id ,\,
 heap \.{:} OKHeap ]}%
\begin{lcom}{101.36}%
\begin{cpar}{0}{F}{F}{0}{0}{}%
For convenience, we assume that evaluating method
 \ensuremath{M} produces some result and new heap for arbitrary
 \ensuremath{OKObject} \ensuremath{obj}, argument list \ensuremath{args}, and
 heap \ensuremath{H} even
 when they are meaningless--for example, if the
 result of executing \ensuremath{M} depends on the values of
 fields of \ensuremath{obj} that do not exist.
\end{cpar}%
\end{lcom}%
\@x{\@s{90.25} \.{\land} \A\, id \.{\in} Id \.{:}}%
 \@x{\@s{108.58} ( H [ id ] \.{\neq} E ( H ) . heap [ id ] ) \.{\implies}
 \.{\lor} id \.{\in} UseableId ( H )}%
\@x{\@s{231.52} \.{\lor} H [ id ] \.{=} {\langle} {\rangle}}%
\begin{lcom}{101.36}%
\begin{cpar}{0}{F}{F}{0}{0}{}%
Evaluating \ensuremath{M} can modify the heap only by modifying
 objects whose \ensuremath{Id} is reachable from \ensuremath{obj} or an
 argument and by creating new objects.
\end{cpar}%
\end{lcom}%
\@x{\@s{70.94} \.{\land} \A\, H1 ,\, H2 \.{\in} OKHeap \.{:}}%
\@x{\@s{90.25} \.{\land} UseableId ( H1 ) \.{=} UseableId ( H2 )}%
 \@x{\@s{90.25} \.{\land} \A\, id \.{\in} UseableId ( H1 ) \.{:} H1 [ id ]
 \.{=} H2 [ id ]}%
 \@x{\@s{90.25} \.{\implies}\@s{4.1} \.{\land} E ( H1 ) . result \.{=} E ( H2
 ) . result}%
 \@x{\@s{109.91} \.{\land} \A\, id \.{\in} Id \.{:} ( H1 [ id ] \.{\neq} E (
 H1 ) . heap [ id ] ) \.{\implies}}%
\@x{\@s{177.11} ( E ( H1 ) . heap [ id ] \.{=} E ( H2 ) . heap [ id ] )}%
\begin{lcom}{86.15}%
\begin{cpar}{0}{F}{F}{0}{0}{}%
If two heaps are the same on the \ensuremath{Ids} reachable from
 \ensuremath{obj} and the arguments, then evaluating \ensuremath{M} in the two
 heaps produces the same result, and it produces the
 same changes to the two heaps.
\end{cpar}%
\end{lcom}%
\@x{}\midbar\@xx{}%
\begin{lcom}{0}%
\begin{cpar}{0}{F}{F}{0}{0}{}%
Actions
\end{cpar}%
\end{lcom}%
\@x{ {\VARIABLE} heap\@s{4.1}}%
\@y{\@s{0}%
 The variable whose value is the current \ensuremath{Heap}.
}%
\@xx{}%
\par\vspace{8.0pt}%
\@x{ ResolveWhenPromise ( id ) \.{\defeq}}%
\begin{lcom}{8.2}%
\begin{cpar}{0}{F}{F}{0}{0}{}%
 The action that modifies the heap by resolving a \ensuremath{when} promise
 when
 the promise it is waiting for has been resolved.
\end{cpar}%
\end{lcom}%
\@x{\@s{12.29} \.{\LET} promise \.{\defeq} heap [ id ]}%
\@x{\@s{12.29} \.{\IN}\@s{4.1} \.{\land} promise \.{\in} Promise}%
\@x{\@s{36.79} \.{\land} {\lnot} promise . resolved}%
\@x{\@s{36.79} \.{\land} promise . isWhen}%
\@x{\@s{36.79} \.{\land} heap [ promise . argPromise ] . resolved}%
\@x{\@s{36.79} \.{\land} heap \.{'} \.{=}}%
\@x{\@s{52.01} \.{\LET} E \.{\defeq} Eval ( promise . method ,\,}%
\@x{\@s{123.25} heap [ promise . objId ] ,\,}%
 \@x{\@s{123.25} {\langle} heap [ promise . argPromise ] . value {\rangle}
 ,\,}%
\@x{\@s{123.25} heap )}%
 \@x{\@s{52.01} \.{\IN}\@s{4.1} [ E . heap {\EXCEPT} {\bang} [ id ] \.{=} [
 type\@s{16.76} \.{\mapsto}\@w{promise} ,\,}%
\@x{\@s{187.57} resolved \.{\mapsto} {\TRUE} ,\,}%
\@x{\@s{187.57} value\@s{11.90} \.{\mapsto} E . result ] ]}%
\par\vspace{8.0pt}%
\@x{ ResolveNonWhenPromise ( id ) \.{\defeq}}%
\begin{lcom}{8.2}%
\begin{cpar}{0}{F}{F}{0}{0}{}%
The action that modifies the heap by resolving a non-\ensuremath{when} promise
 when the promise it is waiting for has been resolved.
\end{cpar}%
\end{lcom}%
\@x{\@s{12.29} \.{\LET} promise \.{\defeq} heap [ id ]}%
\@x{\@s{12.29} \.{\IN}\@s{4.1} \.{\land} promise \.{\in} Promise}%
\@x{\@s{36.79} \.{\land} {\lnot} promise . resolved}%
\@x{\@s{36.79} \.{\land} {\lnot} promise . isWhen}%
\@x{\@s{36.79} \.{\land} heap [ promise . objIdPromise ] . resolved}%
\@x{\@s{36.79} \.{\land} heap \.{'} \.{=}}%
\@x{\@s{52.01} \.{\LET} E \.{\defeq} Eval ( promise . method ,\,}%
\@x{\@s{123.25} heap [ heap [ promise . objIdPromise ] . value ] ,\,}%
\@x{\@s{123.25} promise . args ,\,}%
\@x{\@s{123.25} heap )}%
 \@x{\@s{52.01} \.{\IN}\@s{4.1} [ E . heap {\EXCEPT} {\bang} [ id ] \.{=} [
 type\@s{16.76} \.{\mapsto}\@w{promise} ,\,}%
\@x{\@s{187.57} resolved \.{\mapsto} {\TRUE} ,\,}%
\@x{\@s{187.57} value\@s{11.90} \.{\mapsto} E . result ] ]}%
\par\vspace{8.0pt}%
\@x{ GarbageCollectResolvedPromise ( id ) \.{\defeq}}%
\begin{lcom}{8.2}%
\begin{cpar}{0}{F}{F}{0}{0}{}%
The action that garbage collects a promise that has been resolved and
 whose value is no longer usable.
\end{cpar}%
\end{lcom}%
\@x{\@s{8.2} \.{\land} heap [ id ] \.{\in} Promise}%
\@x{\@s{8.2} \.{\land} heap [ id ] . resolved}%
\@x{\@s{8.2} \.{\land} \A\, i \.{\in} Id \.{\,\backslash\,} \{ id \} \.{:}}%
 \@x{\@s{26.53} ( heap [ i ] \.{\neq} {\langle} {\rangle} ) \.{\implies} \A\,
 fldName \.{\in} {\DOMAIN} i \.{:}}%
\@x{\@s{107.89} ( heap [ i ] [ fldName ] \.{\in} Id ) \.{\implies}}%
\@x{\@s{115.88} ( heap [ heap [ i ] [ fldName ] ] \.{\neq} id )}%
 \@x{\@s{8.2} \.{\land} heap \.{'} \.{=} [ heap {\EXCEPT} {\bang} [ id ] \.{=}
 {\langle} {\rangle} ]}%
\par\vspace{16.0pt}%
\@x{}\bottombar\@xx{}%
\setboolean{shading}{false}
\begin{lcom}{0}%
\begin{cpar}{0}{F}{F}{0}{0}{}%
\ensuremath{\.{\,\backslash\,}}* Modification History
\end{cpar}%
\begin{cpar}{0}{F}{F}{0}{0}{}%
 \ensuremath{\.{\,\backslash\,}}* Last modified Sun \ensuremath{Mar} 06
 11:28:30 \ensuremath{PST} 2011 by \ensuremath{lamport
}%
\end{cpar}%
\begin{cpar}{0}{F}{F}{0}{0}{}%
 \ensuremath{\.{\,\backslash\,}}* Created Sat \ensuremath{Mar} 05 13:24:42
 \ensuremath{PST} 2011 by \ensuremath{lamport
}%
\end{cpar}%
\end{lcom}%
\end{document}
